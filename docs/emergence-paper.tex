\documentclass{article}

\title{Accelerating Large Scale Simulations Using Peer-to-Peer Technology}
\date{2021-02-07}
\author{Ryuusuke Okuno}

\begin{document}

\maketitle
\pagenumbering{gobble}
\newpage
\pagenumbering{arabic}

\section{Abstract}
This paper describes a new way to simulate large systems (i.e. a game
simulating virtual creatures evolving in complex environments) using
distributed computing over a network. The simulation is concurrently
run by many computers connected to each other which calculate a small
part of the complete simulation. It will be shown that it is possible
to create a very large scale simulation using this technique, and
implementation of the ideas described here can be found in the game I
am currently working on.

\section{How To Gain More Computing Power?}
\subsection{The Hardware Run}
Modern video games developers assume that every player has the latest
and greatest technology to play their games, and this forces the
player to buy newer hardware for each game generation. This also
reduces the game to a particular era in which it was popular, making
it a simple fashion thing, until another brand new game comes out with
amazing new graphics and even more hardware requirements than the
previous one. Noticeable exceptions include ``Minecraft'', ``Roblox''
and ``Tetris''. What distinguishes theses games from the enormous pile
of ``Brand new AAA games'' is that they have very low hardware
requirements: Minecraft, for example, only has one element to create
everything from, pixelated blocks.

\subsection{Moore's Law is Getting Weaker}
Moore's Law, which stipulates that the computing power of
microprocessors increases exponentially, is not valid anymore. We have
reached the physical limit where we can't just add more
transistors. Nowadays, CPU manufacturers are just adding more threads
to their microprocessors, not making them more powerful. The hardware
run cannot keep living longer under theses conditions. New games will
need to find a way to be even more realistic and this also applies to
supercomputers: the most powerful supercomputer as I am writing this
is a combination of several computers working concurrently.

\section{Network Architecture}
\subsection{Small World theory}
I think that using modern network architecture, we can achieve
infinite computing power. Modern multiplayer games use a
client-server architecture, which has many flaws.

\end{document}
